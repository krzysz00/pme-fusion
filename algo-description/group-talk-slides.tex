\documentclass{beamer}

\usepackage{mathtools,amsthm,cancel}
\DeclarePairedDelimiter\ceil{\lceil}{\rceil}
\DeclarePairedDelimiter\floor{\lfloor}{\rfloor}

\newcommand*{\mycite}[1]{~\cite{#1}}

\usepackage{hyperref}

\usepackage{biblatex}
\addbibresource{cites.bib}

% 
% A FLAME algorithm is bracketed by
%
%  \begin{FlameAlg}
%    <statements>
%  \end{FlameAlg}

\newenvironment{FlameAlg}{
\begin{tabbing}
in \= in \= in  \= in  \= in  \= \kill
}
{
\end{tabbing}
}

\newboolean{IsWide}
\setboolean{IsWide}{true}

\newenvironment{FlameAlgNarrow}{
\setboolean{IsWide}{false}
\begin{tabbing}
in \= in \= in  \= in  \= in  \= \kill
}
{
\end{tabbing}
\setboolean{IsWide}{true}
}

%
% A loop in a FLAME algorithm is bracketed by
%
% \FlaDoUntil{ <condition> }
%   Body of the loop
% \FlaEndDo
%
% The body is indented

\newcommand{\FlaDoUntil}[1]{
{ \bf do until #1 } \+ 
}

% Note: because of the \kill, it is important
% to have a \\ before the \FlaEndDo
% since otherwise the last line before the
% \FlaEndDo will not show

\newcommand{\FlaEndDo}{
\- \kill
{ \bf enddo }
}

% In math mode, 
% \FlaTwoByTwo{A}{B}
%             {C}{D}
% creates the picture
%   / A || B \
%   | ==  == |
%   \ C || D /

\newcommand{\FlaTwoByTwo}[4]{
\left( 
\begin{array}{c || c}
#1 & #2 \\ \hline \hline
#3 & #4 
\end{array} 
\right)
}

\newcommand{\FlaTwoByTwoSingleLine}[4]{
\left(  
\begin{array}{c | c}
#1 & #2 \\ \hline
#3 & #4 
\end{array} 
\right)
}

% In math mode, 
% \FlaTwoByOne{A}
%             {C}
% creates the picture
%   / A  \
%   | == |
%   \ C  /

\newcommand{\FlaTwoByOne}[2]{
\left( 
\begin{array}{c}
#1 \\ \hline \hline
#2 
\end{array} 
\right)
}

% In math mode, 
% \FlaTwoByOneSingleLine{A}
%                       {C}
% creates the picture
%   / A  \
%   | -- |
%   \ C  /

\newcommand{\FlaTwoByOneSingleLine}[2]{
\left( 
\begin{array}{c}
#1 \\ \hline
#2 
\end{array} 
\right)
}

% In math mode, 
% \FlaOneByTwo{A}{B}
% creates the picture
%   ( A || B )

\newcommand{\FlaOneByTwo}[2]{
\left( 
\begin{array}{c || c}
#1 & #2 
\end{array} 
\right)
}

\newcommand{\FlaOneByTwoSingleLine}[2]{
\left( 
\begin{array}{c | c}
#1 & #2
\end{array} 
\right)
}

% In math mode, 
% \FlaThreeByThreeTL{A}{B}{C}
%                   {D}{E}{F}
%                   {G}{H}{I}
% creates the picture
%   / A | B || C \
%   | -- ---  -- |
%   | D | E || F |
%   | ==  ==  == |
%   \ G | H || I /
% Notice: the TL means that the
% center block (E) is part of the
% TL quadrant, where quadrants are
% partitioned by the double lines.

\newcommand{\FlaThreeByThreeTL}[9]{
\left( 
\begin{array}{c | c || c}
#1 & #2 & #3 \\ \hline
#4 & #5 & #6 \\ \hline \hline 
#7 & #8 & #9
\end{array} 
\right) 
}

% In math mode, 
% \FlaThreeByThreeBR{A}{B}{C}
%                   {D}{E}{F}
%                   {G}{H}{I}
% creates the picture
%   / A || B | C \
%   | ==  ==  == |
%   | D || E | F |
%   | -- ---  -- |
%   \ G || H | I /
% Notice: the BR means that the
% center block (E) is part of the
% BR quadrant, where quadrants are
% partitioned by the double lines.

\newcommand{\FlaThreeByThreeBR}[9]{
\left( 
\begin{array}{c || c | c}
#1 & #2 & #3 \\ \hline \hline 
#4 & #5 & #6 \\ \hline
#7 & #8 & #9
\end{array} 
\right)
}

% In math mode, 
% \FlaOneByThreeR{A}{B}{C}
% creates the picture
%   ( A || B | C )
% Notice: the R means that the
% center block (B) is part of the
% R(ight) submatrix, where 
% submatrices are % partitioned 
% by the double lines.

\newcommand{\FlaOneByThreeR}[3]{
\left( 
\begin{array}{c || c | c}
#1 & #2 & #3 
\end{array} 
\right)
}

% In math mode, 
% \FlaOneByThreeL{A}{B}{C}
% creates the picture
%   ( A | B || C )
% Notice: the R means that the
% center block (B) is part of the
% R(ight) submatrix, where 
% submatrices are % partitioned 
% by the double lines.

\newcommand{\FlaOneByThreeL}[3]{
\left( 
\begin{array}{c | c || c}
#1 & #2 & #3 
\end{array} 
\right)
}

% In math mode, 
% \FlaThreeByOneT{A}
%                {D}
%                {G}
% creates the picture
%   / A  \
%   | == |
%   | B  |
%   | -- |
%   \ C  /
% Notice: the T means that the
% center block (C) is part of the
% T(op) submatrix where submatrices
% are % partitioned by the double 
% lines.

\newcommand{\FlaThreeByOneT}[3]{
\left( 
\begin{array}{c}
#1 \\ \hline
#2 \\ \hline \hline 
#3 
\end{array} 
\right) 
}

% In math mode, 
% \FlaThreeByOneB{A}
%                {D}
%                {G}
% creates the picture
%   / A  \
%   | -- |
%   | B  |
%   | == |
%   \ C  /
% Notice: the B means that the
% center block (C) is part of the
% T(op) submatrix where submatrices
% are % partitioned by the double 
% lines.

\newcommand{\FlaThreeByOneB}[3]{
\left( 
\begin{array}{c}
#1 \\ \hline \hline 
#2 \\ \hline
#3 
\end{array} 
\right) 
}

% Various key words

% The following is a typical use of 
% \FlaPartition:
% 
% \FlaPartition {
%    $ A \rightarrow \FlaTwoByTwo{ A_{TL} }{ A_{TR} }
%                                { A_{BL} }{ A_{BR} } $ 
% }
% {
%    where $ A_{TL} $ is $ 0 \times 0 $
% }
% 
% Creates something like
% Partition A -> / A_TL || A_TR \
%                | =====  ===== |
%                \ A_BL || A_BR /
% where A_TL is 0 x 0
%

\newcommand{\FlaPartition}[2]{
\ifthenelse{\boolean{IsWide}}{
{\bf partition } \hspace{-1em} #1 \hspace{-1em} #2 
}
{ 
{\bf partition } \+ \\ #1 \+ \\ #2 \- \-
}
}

% The following is a typical use of 
% \FlaRepartition:
% 
% \FlaRepartition{
% $ 
% \FlaTwoByTwo{ A_{TL} }{ A_{TR} }
%             { A_{BL} }{ A_{BR} } \rightarrow
% \FlaThreeByThreeBR{ A_{00} }{ A_{01} }{ A_{02} }
%                   { A_{10} }{ A_{11} }{ A_{12} }
%                   { A_{20} }{ A_{21} }{ A_{22} } 
% $ 
% }
% {
%    \FlaWhere{$ A_{11} $ is $ b \times b $}
%  }
% 
% Creates something like
% Repartition  
%   / A_TL || A_TR \    / A_00 || A_01 | A_02 \
%   | =====  ===== | -> | =====  ====== ===== |
%   \ A_BL || A_BR /    | A_10 || A_11 | A_12 |
%                       | -----  ------ ----- |
%                       \ A_20 || A_21 | A_22 /
% where A_11 is b x b
%

\newcommand{\FlaRepartition}[2]{
\ifthenelse{\boolean{IsWide}}{
\hspace{-8pt}
{\bf repartition } \hspace{-1em} #1 \hspace{-1em} #2 
}
{
\hspace{-8pt}
{\bf repartition } \+ \\ #1 \+ \\ #2 \- \- 
}
}
% The following is a typical use of 
% \FlaContinue:
% 
% \FlaContinue{
% $ 
%    \FlaTwoByTwo{ A_{TL} }{ A_{TR} }
%                { A_{BL} }{ A_{BR} } \leftarrow
%    \FlaThreeByThreeTL{ A_{00} }{ A_{01} }{ A_{02} }
%                      { A_{10} }{ A_{11} }{ A_{12} }
%                      { A_{20} }{ A_{21} }{ A_{22} } 
% $ 
% }
% 
% Creates something like
% Continue with
%   / A_TL || A_TR \    / A_00 | A_01 || A_02 \
%   | =====  ===== | <- | ----- ------  ----- |
%   \ A_BL || A_BR /    | A_10 | A_11 || A_12 |
%                       | ===== ======  ===== |
%                       \ A_20 | A_21 || A_22 /

\newcommand{\FlaContinue}[1]{
\ifthenelse{\boolean{IsWide}}{
\hspace{-8pt}
{\bf continue with } #1
}
{
\hspace{-8pt}
{\bf continue with } \+ \\ #1 \-
}
}
\newcommand{\FlaWhere}[1]{
\hspace{-1em} { \bf where #1 } 
}

\newcommand{\undetermined}{ \star }

\newcommand{\FlaStartCompute}{
\setlength{\unitlength}{0.95in}
\begin{picture}(3,0.01)
\put(0,0){\line(1,0){3}}
\put(0,0.01){\line(1,0){3}}
\end{picture} 
}

\newcommand{\FlaEndCompute}{
\setlength{\unitlength}{0.95in}
\begin{picture}(3,0.01)
\put(0,0){\line(1,0){3}} 
\put(0,0.01){\line(1,0){3}} 
\end{picture} 
}

\newcommand{\FlaUpLo}[2]{
{#1} \backslash {#2}
}

\newcommand{\FlaInverse}[1]{
{ #1 }^{-1}
}



\newcommand*{\opF}{\mathcal{F}}
\newcommand*{\opf}{f}
\def\?#1{}

\useoutertheme{infolines}
\setbeamertemplate{navigation symbols}{}

\title[Loop fusion]{Automated High-Level Loop Fusion for FLAME Algorithms}
\author[Drewniak]{Krzysztof A. Drewniak}
\institute[CMU]{Carnegie Mellon University}
\date[]{June TODO, 2018}

\begin{document}
\begin{frame}[plain]
  \titlepage{}
\end{frame}

\begin{frame}
  \frametitle{High-level loop fusion}
  \begin{itemize}
  \item Problems often are a series of subproblems
  \item Combining subalgorithms often helps performance
  \item Goal: find all the fused algorithms for a problem
  \item Compilers know too many details - need a high level approach
  \end{itemize}
\end{frame}

\begin{frame}
  \frametitle{FLAME algorithms, loop invariants}
  \begin{itemize}
  \item FLAME = Formal Linear Algebra Methods Eenvironments
  \item Provably correct algorithms from spec
  \item Algorithms $\Leftrightarrow$ loop invariants
  \item We know how to:
    \begin{itemize}
    \item Autogenerate algorithm/code from loop invariant
    \item Autogenerate all possible loop invariants
    \item Identify when fusion is possible (in theory)
    \end{itemize}
  \end{itemize}
\end{frame}

\begin{frame}
  \frametitle{What we add}
  \begin{itemize}
  \item Autogenerate all sets of fusable loop invariants
  \item Input is \emph{partitioned matrix expression} --- indicates needed computations
  \item Can be used to generate code
  \end{itemize}
\end{frame}

\section*{FLAME}

\begin{frame}
  \frametitle{Goal}
  Want to compute
  \begin{columns}
    \begin{column}{0.5\textwidth}
      \begin{equation*}
        \widetilde{A} = \opF(\hat(A), \underbrace{\ldots}_{O})
      \end{equation*}
    \end{column}
    \begin{column}{0.5\textwidth}
      \begin{equation*}
        \widetilde{A} = CHOL(\hat{A})
      \end{equation*}
    \end{column}
  \end{columns}

  $\hat{A}$ and $\widetilde{A}$ share memory ($A$).

  Initially, $A = \hat{A}$.

  At termination, $A = \widetilde{A}$.
\end{frame}

\begin{frame}
  \frametitle{Algorithm structure}
  \begin{FlameAlg}
    \FlaPartition{ $A \rightarrow \FlaTwoByTwo{A_{TL}}{A_{TR}}{A_{BL}}{A_{BR}}$}{\\
      $\quad$ where $\operatorname{dim}(A_{TL}) = 0 \times 0$}\\
    \FlaDoUntil{ $\operatorname{dim}(A_{TL}) = n \times n$}\\
    \FlaRepartition{$\FlaTwoByTwo{A_{TL}}{A_{TR}}{A_{BL}}{A_{BR}}%
      \rightarrow \FlaThreeByThreeBR{A_{00}}{a_{01}}{A_{02}}%
      {a_{10}^T}{\alpha_{11}}{a_{12}^T}%
      {A_{02}}{a_{21}}{A_{22}}$}\\
    $\?[\vdots]\text{ loop body}$\\ % Make the balanced braces check shut up
    \FlaContinue{$\FlaTwoByTwo{A_{TL}}{A_{TR}}{A_{BL}}{A_{BR}} \leftarrow{}%
      \FlaThreeByThreeTL{A_{00}}{a_{01}}{A_{02}}%
      {a_{10}^T}{\alpha_{11}}{a_{12}^T}%
      {A_{20}}{a_{21}}{A_{22}}$}\\
    \FlaEndDo{}
  \end{FlameAlg}
\end{frame}

\begin{frame}
  \frametitle{Algoriithm example}
  \begin{FlameAlg}
    \FlaPartition{ $A \rightarrow \FlaTwoByTwo{A_{TL}}{*}{A_{BL}}{A_{BR}}$}{\\
      $\quad$ where $\operatorname{dim}(A_{TL}) = 0 \times 0$}\\
    \FlaDoUntil{ $\operatorname{dim}(A_{TL}) = n \times n$}\\
    \FlaRepartition{$\FlaTwoByTwo{A_{TL}}{*}{A_{BL}}{A_{BR}}%
      \rightarrow \FlaThreeByThreeBR{A_{00}}{*}{*}%
      {a_{10}^T}{\alpha_{11}}{*}%
      {A_{02}}{a_{21}}{A_{22}}$}\\
    $\alpha_{11} \coloneqq \sqrt{\alpha_{11}}$\\
    $a_{21} \coloneqq a_{21} / \alpha_{11}$\\
    $A_{22} \coloneqq A_{22} - a_{21}a_{21}^T$\\
    \FlaContinue{$\FlaTwoByTwo{A_{TL}}{*}{A_{BL}}{A_{BR}} \leftarrow{}%
      \FlaThreeByThreeTL{A_{00}}{*}{*}%
      {a_{10}^T}{\alpha_{11}}{*}%
      {A_{20}}{a_{21}}{A_{22}}$}\\
    \FlaEndDo{}
  \end{FlameAlg}
\end{frame}

\begin{frame}
  \frametitle{Partitioned Matrix Expressions}
  \begin{itemize}
  \item Take $A$ (and maybe other stuff), split it into regions.
  \item Lines between regions move during algorithm
  \end{itemize}
  \begin{equation*}
    \FlaTwoByTwo{\widetilde{A}_{TL} = \opF_{TL}(\hat{A}, \ldots)}{\widetilde{A}_{TR} = \opF_{TR}(\hat{A}, \ldots)}
    {\widetilde{A}_{BL} = \opF_{BL}(\hat{A}, \ldots)}{\widetilde{A}_{BR} = \opF_{BR}(\hat{A}, \ldots)}
  \end{equation*}

  \begin{equation*}
    \FlaTwoByTwo{\widetilde{A}_{TL} = CHOL(\hat{A}_{TL})}{*}
    {\widetilde{A}_{BL} = \hat{A}_{BL}\widetilde{A}_{TL}^{-T}}
    {\widetilde{A}_{BR} = CHOL(\hat{A}_{BR} - \widetilde{A}_{BL}\widetilde{A}_{BL}^T)}
  \end{equation*}
\end{frame}

\begin{frame}
  \frametitle{Loop invariants}
  \begin{itemize}
  \item Find $\opf_R$ and $\opf'_R$ so $\opF_R(\hat{A}) = \opf'(\opf(\hat{A}))$.
  \item $\opf_R$ is loop invariant for $R$, $\opf'_R$ is remainder
  \item Invariant for algorithm is an invariant per region
  \item Completely determine algorithm
  \end{itemize}
\end{frame}

\begin{frame}
  \frametitle{This is a loop invariant}
  Starting from Cholesky's PME:
  \begin{equation*}
    \FlaTwoByTwo{\widetilde{A}_{TL} = CHOL(\hat{A}_{TL})}{*}
    {\widetilde{A}_{BL} = \hat{A}_{BL}\widetilde{A}_{TL}^{-T}}
    {\widetilde{A}_{BR} = CHOL(\hat{A}_{BR} - \widetilde{A}_{BL}\widetilde{A}_{BL}^T)}
  \end{equation*}

  We obtain
  \begin{equation*}
    \FlaTwoByTwo{A_{TL} = CHOL(\hat{A}_{TL})}{*}
    {A_{BL} = \hat{A}_{BL}\widetilde{A}_{TL}^{-T}}
    {A_{BR} = \hat{A}_{BR} - \widetilde{A}_{BL}\widetilde{A}_{BL}^T}
  \end{equation*}
\end{frame}

\begin{frame}
  \frametitle{As are these}
  \begin{equation*}
    \FlaTwoByTwo{A_{TL} = CHOL(\hat{A}_{TL})}{*}
    {A_{BL} = \hat{A}_{BL}\widetilde{A}_{TL}^{-T}}
    {A_{BR} = \hat{A}_{BR}}
  \end{equation*}

  \begin{equation*}
    \FlaTwoByTwo{A_{TL} = CHOL(\hat{A}_{TL})}{*}
    {A_{BL} = \hat{A}_{BL}}
    {A_{BR} = \hat{A}_{BR}}
  \end{equation*}
\end{frame}

\begin{frame}
  \frametitle{But not these}
  \begin{equation*}
    \FlaTwoByTwo{A_{TL} = CHOL(\hat{A}_{TL})}{*}
    {A_{BL} = \hat{A}_{BL}\widetilde{A}_{TL}^{-T}}
    {A_{BR} = CHOL(\hat{A}_{BR} - \widetilde{A}_{BL}\widetilde{A}_{BL}^T)}
  \end{equation*}

  \begin{equation*}
    \FlaTwoByTwo{A_{TL} = \hat{A}_{TL}}{*}
    {A_{BL} = \hat{A}_{BL}}
    {A_{BR} = \hat{A}_{BR}}
  \end{equation*}
\end{frame}

\begin{frame}
  \frametitle{Or this}
  \begin{equation*}
    \FlaTwoByTwo{A_{TL} = \hat{A}_{TL}}{*}
    {A_{BL} = \hat{A}_{BL}\widetilde{A}_{TL}^{-T}}
    {A_{BR} = \hat{A}_{BR}}
  \end{equation*}
\end{frame}


\end{document}
