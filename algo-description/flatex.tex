% 
% A FLAME algorithm is bracketed by
%
%  \begin{FlameAlg}
%    <statements>
%  \end{FlameAlg}

\newenvironment{FlameAlg}{
\begin{tabbing}
in \= in \= in  \= in  \= in  \= \kill
}
{
\end{tabbing}
}

\newboolean{IsWide}
\setboolean{IsWide}{true}

\newenvironment{FlameAlgNarrow}{
\setboolean{IsWide}{false}
\begin{tabbing}
in \= in \= in  \= in  \= in  \= \kill
}
{
\end{tabbing}
\setboolean{IsWide}{true}
}

%
% A loop in a FLAME algorithm is bracketed by
%
% \FlaDoUntil{ <condition> }
%   Body of the loop
% \FlaEndDo
%
% The body is indented

\newcommand{\FlaDoUntil}[1]{
{ \bf do until #1 } \+ 
}

% Note: because of the \kill, it is important
% to have a \\ before the \FlaEndDo
% since otherwise the last line before the
% \FlaEndDo will not show

\newcommand{\FlaEndDo}{
\- \kill
{ \bf enddo }
}

% In math mode, 
% \FlaTwoByTwo{A}{B}
%             {C}{D}
% creates the picture
%   / A || B \
%   | ==  == |
%   \ C || D /

\newcommand{\FlaTwoByTwo}[4]{
\left( 
\begin{array}{c || c}
#1 & #2 \\ \hline \hline
#3 & #4 
\end{array} 
\right)
}

\newcommand{\FlaTwoByTwoSingleLine}[4]{
\left(  
\begin{array}{c | c}
#1 & #2 \\ \hline
#3 & #4 
\end{array} 
\right)
}

% In math mode, 
% \FlaTwoByOne{A}
%             {C}
% creates the picture
%   / A  \
%   | == |
%   \ C  /

\newcommand{\FlaTwoByOne}[2]{
\left( 
\begin{array}{c}
#1 \\ \hline \hline
#2 
\end{array} 
\right)
}

% In math mode, 
% \FlaTwoByOneSingleLine{A}
%                       {C}
% creates the picture
%   / A  \
%   | -- |
%   \ C  /

\newcommand{\FlaTwoByOneSingleLine}[2]{
\left( 
\begin{array}{c}
#1 \\ \hline
#2 
\end{array} 
\right)
}

% In math mode, 
% \FlaOneByTwo{A}{B}
% creates the picture
%   ( A || B )

\newcommand{\FlaOneByTwo}[2]{
\left( 
\begin{array}{c || c}
#1 & #2 
\end{array} 
\right)
}

\newcommand{\FlaOneByTwoSingleLine}[2]{
\left( 
\begin{array}{c | c}
#1 & #2
\end{array} 
\right)
}

% In math mode, 
% \FlaThreeByThreeTL{A}{B}{C}
%                   {D}{E}{F}
%                   {G}{H}{I}
% creates the picture
%   / A | B || C \
%   | -- ---  -- |
%   | D | E || F |
%   | ==  ==  == |
%   \ G | H || I /
% Notice: the TL means that the
% center block (E) is part of the
% TL quadrant, where quadrants are
% partitioned by the double lines.

\newcommand{\FlaThreeByThreeTL}[9]{
\left( 
\begin{array}{c | c || c}
#1 & #2 & #3 \\ \hline
#4 & #5 & #6 \\ \hline \hline 
#7 & #8 & #9
\end{array} 
\right) 
}

% In math mode, 
% \FlaThreeByThreeBR{A}{B}{C}
%                   {D}{E}{F}
%                   {G}{H}{I}
% creates the picture
%   / A || B | C \
%   | ==  ==  == |
%   | D || E | F |
%   | -- ---  -- |
%   \ G || H | I /
% Notice: the BR means that the
% center block (E) is part of the
% BR quadrant, where quadrants are
% partitioned by the double lines.

\newcommand{\FlaThreeByThreeBR}[9]{
\left( 
\begin{array}{c || c | c}
#1 & #2 & #3 \\ \hline \hline 
#4 & #5 & #6 \\ \hline
#7 & #8 & #9
\end{array} 
\right)
}

% In math mode, 
% \FlaOneByThreeR{A}{B}{C}
% creates the picture
%   ( A || B | C )
% Notice: the R means that the
% center block (B) is part of the
% R(ight) submatrix, where 
% submatrices are % partitioned 
% by the double lines.

\newcommand{\FlaOneByThreeR}[3]{
\left( 
\begin{array}{c || c | c}
#1 & #2 & #3 
\end{array} 
\right)
}

% In math mode, 
% \FlaOneByThreeL{A}{B}{C}
% creates the picture
%   ( A | B || C )
% Notice: the R means that the
% center block (B) is part of the
% R(ight) submatrix, where 
% submatrices are % partitioned 
% by the double lines.

\newcommand{\FlaOneByThreeL}[3]{
\left( 
\begin{array}{c | c || c}
#1 & #2 & #3 
\end{array} 
\right)
}

% In math mode, 
% \FlaThreeByOneT{A}
%                {D}
%                {G}
% creates the picture
%   / A  \
%   | == |
%   | B  |
%   | -- |
%   \ C  /
% Notice: the T means that the
% center block (C) is part of the
% T(op) submatrix where submatrices
% are % partitioned by the double 
% lines.

\newcommand{\FlaThreeByOneT}[3]{
\left( 
\begin{array}{c}
#1 \\ \hline
#2 \\ \hline \hline 
#3 
\end{array} 
\right) 
}

% In math mode, 
% \FlaThreeByOneB{A}
%                {D}
%                {G}
% creates the picture
%   / A  \
%   | -- |
%   | B  |
%   | == |
%   \ C  /
% Notice: the B means that the
% center block (C) is part of the
% T(op) submatrix where submatrices
% are % partitioned by the double 
% lines.

\newcommand{\FlaThreeByOneB}[3]{
\left( 
\begin{array}{c}
#1 \\ \hline \hline 
#2 \\ \hline
#3 
\end{array} 
\right) 
}

% Various key words

% The following is a typical use of 
% \FlaPartition:
% 
% \FlaPartition {
%    $ A \rightarrow \FlaTwoByTwo{ A_{TL} }{ A_{TR} }
%                                { A_{BL} }{ A_{BR} } $ 
% }
% {
%    where $ A_{TL} $ is $ 0 \times 0 $
% }
% 
% Creates something like
% Partition A -> / A_TL || A_TR \
%                | =====  ===== |
%                \ A_BL || A_BR /
% where A_TL is 0 x 0
%

\newcommand{\FlaPartition}[2]{
\ifthenelse{\boolean{IsWide}}{
{\bf partition } \hspace{-1em} #1 \hspace{-1em} #2 
}
{ 
{\bf partition } \+ \\ #1 \+ \\ #2 \- \-
}
}

% The following is a typical use of 
% \FlaRepartition:
% 
% \FlaRepartition{
% $ 
% \FlaTwoByTwo{ A_{TL} }{ A_{TR} }
%             { A_{BL} }{ A_{BR} } \rightarrow
% \FlaThreeByThreeBR{ A_{00} }{ A_{01} }{ A_{02} }
%                   { A_{10} }{ A_{11} }{ A_{12} }
%                   { A_{20} }{ A_{21} }{ A_{22} } 
% $ 
% }
% {
%    \FlaWhere{$ A_{11} $ is $ b \times b $}
%  }
% 
% Creates something like
% Repartition  
%   / A_TL || A_TR \    / A_00 || A_01 | A_02 \
%   | =====  ===== | -> | =====  ====== ===== |
%   \ A_BL || A_BR /    | A_10 || A_11 | A_12 |
%                       | -----  ------ ----- |
%                       \ A_20 || A_21 | A_22 /
% where A_11 is b x b
%

\newcommand{\FlaRepartition}[2]{
\ifthenelse{\boolean{IsWide}}{
\hspace{-8pt}
{\bf repartition } \hspace{-1em} #1 \hspace{-1em} #2 
}
{
\hspace{-8pt}
{\bf repartition } \+ \\ #1 \+ \\ #2 \- \- 
}
}
% The following is a typical use of 
% \FlaContinue:
% 
% \FlaContinue{
% $ 
%    \FlaTwoByTwo{ A_{TL} }{ A_{TR} }
%                { A_{BL} }{ A_{BR} } \leftarrow
%    \FlaThreeByThreeTL{ A_{00} }{ A_{01} }{ A_{02} }
%                      { A_{10} }{ A_{11} }{ A_{12} }
%                      { A_{20} }{ A_{21} }{ A_{22} } 
% $ 
% }
% 
% Creates something like
% Continue with
%   / A_TL || A_TR \    / A_00 | A_01 || A_02 \
%   | =====  ===== | <- | ----- ------  ----- |
%   \ A_BL || A_BR /    | A_10 | A_11 || A_12 |
%                       | ===== ======  ===== |
%                       \ A_20 | A_21 || A_22 /

\newcommand{\FlaContinue}[1]{
\ifthenelse{\boolean{IsWide}}{
\hspace{-8pt}
{\bf continue with } #1
}
{
\hspace{-8pt}
{\bf continue with } \+ \\ #1 \-
}
}
\newcommand{\FlaWhere}[1]{
\hspace{-1em} { \bf where #1 } 
}

\newcommand{\undetermined}{ \star }

\newcommand{\FlaStartCompute}{
\setlength{\unitlength}{0.95in}
\begin{picture}(3,0.01)
\put(0,0){\line(1,0){3}}
\put(0,0.01){\line(1,0){3}}
\end{picture} 
}

\newcommand{\FlaEndCompute}{
\setlength{\unitlength}{0.95in}
\begin{picture}(3,0.01)
\put(0,0){\line(1,0){3}} 
\put(0,0.01){\line(1,0){3}} 
\end{picture} 
}

\newcommand{\FlaUpLo}[2]{
{#1} \backslash {#2}
}

\newcommand{\FlaInverse}[1]{
{ #1 }^{-1}
}

